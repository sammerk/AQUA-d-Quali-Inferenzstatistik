% Options for packages loaded elsewhere
\PassOptionsToPackage{unicode}{hyperref}
\PassOptionsToPackage{hyphens}{url}
\PassOptionsToPackage{dvipsnames,svgnames,x11names}{xcolor}
%
\documentclass[
  letterpaper,
  DIV=11]{scrreprt}

\usepackage{amsmath,amssymb}
\usepackage{iftex}
\ifPDFTeX
  \usepackage[T1]{fontenc}
  \usepackage[utf8]{inputenc}
  \usepackage{textcomp} % provide euro and other symbols
\else % if luatex or xetex
  \usepackage{unicode-math}
  \defaultfontfeatures{Scale=MatchLowercase}
  \defaultfontfeatures[\rmfamily]{Ligatures=TeX,Scale=1}
\fi
\usepackage{lmodern}
\ifPDFTeX\else  
    % xetex/luatex font selection
\fi
% Use upquote if available, for straight quotes in verbatim environments
\IfFileExists{upquote.sty}{\usepackage{upquote}}{}
\IfFileExists{microtype.sty}{% use microtype if available
  \usepackage[]{microtype}
  \UseMicrotypeSet[protrusion]{basicmath} % disable protrusion for tt fonts
}{}
\makeatletter
\@ifundefined{KOMAClassName}{% if non-KOMA class
  \IfFileExists{parskip.sty}{%
    \usepackage{parskip}
  }{% else
    \setlength{\parindent}{0pt}
    \setlength{\parskip}{6pt plus 2pt minus 1pt}}
}{% if KOMA class
  \KOMAoptions{parskip=half}}
\makeatother
\usepackage{xcolor}
\usepackage{svg}
\setlength{\emergencystretch}{3em} % prevent overfull lines
\setcounter{secnumdepth}{5}
% Make \paragraph and \subparagraph free-standing
\ifx\paragraph\undefined\else
  \let\oldparagraph\paragraph
  \renewcommand{\paragraph}[1]{\oldparagraph{#1}\mbox{}}
\fi
\ifx\subparagraph\undefined\else
  \let\oldsubparagraph\subparagraph
  \renewcommand{\subparagraph}[1]{\oldsubparagraph{#1}\mbox{}}
\fi


\providecommand{\tightlist}{%
  \setlength{\itemsep}{0pt}\setlength{\parskip}{0pt}}\usepackage{longtable,booktabs,array}
\usepackage{calc} % for calculating minipage widths
% Correct order of tables after \paragraph or \subparagraph
\usepackage{etoolbox}
\makeatletter
\patchcmd\longtable{\par}{\if@noskipsec\mbox{}\fi\par}{}{}
\makeatother
% Allow footnotes in longtable head/foot
\IfFileExists{footnotehyper.sty}{\usepackage{footnotehyper}}{\usepackage{footnote}}
\makesavenoteenv{longtable}
\usepackage{graphicx}
\makeatletter
\def\maxwidth{\ifdim\Gin@nat@width>\linewidth\linewidth\else\Gin@nat@width\fi}
\def\maxheight{\ifdim\Gin@nat@height>\textheight\textheight\else\Gin@nat@height\fi}
\makeatother
% Scale images if necessary, so that they will not overflow the page
% margins by default, and it is still possible to overwrite the defaults
% using explicit options in \includegraphics[width, height, ...]{}
\setkeys{Gin}{width=\maxwidth,height=\maxheight,keepaspectratio}
% Set default figure placement to htbp
\makeatletter
\def\fps@figure{htbp}
\makeatother
% definitions for citeproc citations
\NewDocumentCommand\citeproctext{}{}
\NewDocumentCommand\citeproc{mm}{%
  \begingroup\def\citeproctext{#2}\cite{#1}\endgroup}
\makeatletter
 % allow citations to break across lines
 \let\@cite@ofmt\@firstofone
 % avoid brackets around text for \cite:
 \def\@biblabel#1{}
 \def\@cite#1#2{{#1\if@tempswa , #2\fi}}
\makeatother
\newlength{\cslhangindent}
\setlength{\cslhangindent}{1.5em}
\newlength{\csllabelwidth}
\setlength{\csllabelwidth}{3em}
\newenvironment{CSLReferences}[2] % #1 hanging-indent, #2 entry-spacing
 {\begin{list}{}{%
  \setlength{\itemindent}{0pt}
  \setlength{\leftmargin}{0pt}
  \setlength{\parsep}{0pt}
  % turn on hanging indent if param 1 is 1
  \ifodd #1
   \setlength{\leftmargin}{\cslhangindent}
   \setlength{\itemindent}{-1\cslhangindent}
  \fi
  % set entry spacing
  \setlength{\itemsep}{#2\baselineskip}}}
 {\end{list}}
\usepackage{calc}
\newcommand{\CSLBlock}[1]{\hfill\break\parbox[t]{\linewidth}{\strut\ignorespaces#1\strut}}
\newcommand{\CSLLeftMargin}[1]{\parbox[t]{\csllabelwidth}{\strut#1\strut}}
\newcommand{\CSLRightInline}[1]{\parbox[t]{\linewidth - \csllabelwidth}{\strut#1\strut}}
\newcommand{\CSLIndent}[1]{\hspace{\cslhangindent}#1}

\KOMAoption{captions}{tableheading}
\makeatletter
\@ifpackageloaded{bookmark}{}{\usepackage{bookmark}}
\makeatother
\makeatletter
\@ifpackageloaded{caption}{}{\usepackage{caption}}
\AtBeginDocument{%
\ifdefined\contentsname
  \renewcommand*\contentsname{Inhaltsverzeichnis}
\else
  \newcommand\contentsname{Inhaltsverzeichnis}
\fi
\ifdefined\listfigurename
  \renewcommand*\listfigurename{Abbildungsverzeichnis}
\else
  \newcommand\listfigurename{Abbildungsverzeichnis}
\fi
\ifdefined\listtablename
  \renewcommand*\listtablename{Tabellenverzeichnis}
\else
  \newcommand\listtablename{Tabellenverzeichnis}
\fi
\ifdefined\figurename
  \renewcommand*\figurename{Abbildung}
\else
  \newcommand\figurename{Abbildung}
\fi
\ifdefined\tablename
  \renewcommand*\tablename{Tabelle}
\else
  \newcommand\tablename{Tabelle}
\fi
}
\@ifpackageloaded{float}{}{\usepackage{float}}
\floatstyle{ruled}
\@ifundefined{c@chapter}{\newfloat{codelisting}{h}{lop}}{\newfloat{codelisting}{h}{lop}[chapter]}
\floatname{codelisting}{Listing}
\newcommand*\listoflistings{\listof{codelisting}{Listingverzeichnis}}
\makeatother
\makeatletter
\makeatother
\makeatletter
\@ifpackageloaded{caption}{}{\usepackage{caption}}
\@ifpackageloaded{subcaption}{}{\usepackage{subcaption}}
\makeatother
\makeatletter
\@ifpackageloaded{fontawesome5}{}{\usepackage{fontawesome5}}
\makeatother
\ifLuaTeX
\usepackage[bidi=basic]{babel}
\else
\usepackage[bidi=default]{babel}
\fi
\babelprovide[main,import]{ngerman}
% get rid of language-specific shorthands (see #6817):
\let\LanguageShortHands\languageshorthands
\def\languageshorthands#1{}
\ifLuaTeX
  \usepackage{selnolig}  % disable illegal ligatures
\fi
\IfFileExists{bookmark.sty}{\usepackage{bookmark}}{\usepackage{hyperref}}
\IfFileExists{xurl.sty}{\usepackage{xurl}}{} % add URL line breaks if available
\urlstyle{same} % disable monospaced font for URLs
\hypersetup{
  pdftitle={AQUA-d-Quali: Inferenzstatistik},
  pdfauthor={Samuel Merk},
  pdflang={de},
  colorlinks=true,
  linkcolor={blue},
  filecolor={Maroon},
  citecolor={Blue},
  urlcolor={Blue},
  pdfcreator={LaTeX via pandoc}}

\title{AQUA-d-Quali: Inferenzstatistik}
\author{Samuel Merk}
\date{5. Januar 2024}

\begin{document}
\maketitle

\renewcommand*\contentsname{Inhaltsverzeichnis}
{
\hypersetup{linkcolor=}
\setcounter{tocdepth}{2}
\tableofcontents
}
\bookmarksetup{startatroot}

\chapter*{Übersicht}\label{uxfcbersicht}
\addcontentsline{toc}{chapter}{Übersicht}

\markboth{Übersicht}{Übersicht}

\section*{Herzlich Willkommen zur Sitzung »Inferenzstatistik«
👋!}\label{herzlich-willkommen-zur-sitzung-inferenzstatistik}
\addcontentsline{toc}{section}{Herzlich Willkommen zur Sitzung
»Inferenzstatistik« 👋!}

\markright{Herzlich Willkommen zur Sitzung »Inferenzstatistik« 👋!}

Heute wollen wir in die Welt der Inferenzstatistik eintauchen. Wie man
in der Kapitelleiste links sehen kann werden dazu die folgenden Dinge
behandeln bzw. üben:

\begin{itemize}
\tightlist
\item
  Die Unterscheidung von Inferenz- und Deskriptivstatistik in
  alltäglicher und wissenschaftlicher Sprache
  \href{Inferenz_versus_Deskriptivstatistik.qmd}{\faIcon{square-up-right}}
\item
  p-Werte \href{p-Werte.qmd}{\faIcon{square-up-right}}

  \begin{itemize}
  \tightlist
  \item
    Interpretation
  \item
    Do's and don'ts
  \end{itemize}
\item
  Bayes Factors \href{Bayes\%20Factors.qmd}{\faIcon{square-up-right}}

  \begin{itemize}
  \tightlist
  \item
    Interpretation
  \item
    Do's and don'ts\\
  \end{itemize}
\item
  Konfidenzintervalle
  \href{Konfidenzintervalle.qmd}{\faIcon{square-up-right}}

  \begin{itemize}
  \tightlist
  \item
    Interpretation
  \item
    Do's and don'ts
  \end{itemize}
\item
  Highest Density Posterior Intervals
  \href{Highest\%20Density\%20Posterior\%20Intervals.qmd}{\faIcon{square-up-right}}

  \begin{itemize}
  \tightlist
  \item
    Interpretation
  \item
    Do's and don'ts
  \end{itemize}
\end{itemize}

\bookmarksetup{startatroot}

\chapter{Inferenz versus
Deskriptivstatistik}\label{sec-Inferenz_versus_Deskriptivstatistik}

\section{Übung}\label{uxfcbung}

Im folgenden findet ihr leicht abgeänderte Pressemitteilungen und sollt
entscheiden ob in dieser eine Inferenz- oder Deskriptivstatistik
präsentiert wird.

\subsection*{Steigerung der Motivation}\label{steigerung-der-motivation}
\addcontentsline{toc}{subsection}{Steigerung der Motivation}

\begin{quote}
»Die Präsentation von Interviewausschnitten zeigte einen signifikanten
Effekt auf die Wertschätzung des Faches Mathematik.«
\end{quote}

Hier wird eine

\begin{itemize}
\item
  \begin{enumerate}
  \def\labelenumi{(\Alph{enumi})}
  \tightlist
  \item
    Deskriptivstatistik\\
  \end{enumerate}
\item
  \begin{enumerate}
  \def\labelenumi{(\Alph{enumi})}
  \setcounter{enumi}{1}
  \tightlist
  \item
    Inferenzstatistik
  \end{enumerate}
\end{itemize}

präsentiert

\subsection*{Faktenboxen und
Risikowahrnehmung}\label{faktenboxen-und-risikowahrnehmung}
\addcontentsline{toc}{subsection}{Faktenboxen und Risikowahrnehmung}

\begin{quote}
»Wenn man annimmt, dass die Faktenboxen die Korrektheit der
Risikoeinschätzung steigern, sind die erhobenen Daten mindestens 20fach
wahrscheinlicher, als wenn man annimmt, dass die Korrektheit der
Risikoeinschätzung gleichbleibt.«
\end{quote}

Hier wird eine

\begin{itemize}
\item
  \begin{enumerate}
  \def\labelenumi{(\Alph{enumi})}
  \tightlist
  \item
    Deskriptivstatistik\\
  \end{enumerate}
\item
  \begin{enumerate}
  \def\labelenumi{(\Alph{enumi})}
  \setcounter{enumi}{1}
  \tightlist
  \item
    Inferenzstatistik
  \end{enumerate}
\end{itemize}

präsentiert

\subsection*{Active Retrieval}\label{active-retrieval}
\addcontentsline{toc}{subsection}{Active Retrieval}

\begin{quote}
»Das aktive Wiederaufrufen von Infromationen via gefaltetem Tandembogen,
war dem wiederholten Lesen des Bogen ergab ein Cohen's d von \(d\) =.2.
Dabei ist Cohen's \(d\) als Differenz der zwei Gruppenmittelwerte
geteilt durch die gemittelte Standardabweichung definiert.«
\end{quote}

\subsection*{Mathematikkompetenz}\label{mathematikkompetenz}
\addcontentsline{toc}{subsection}{Mathematikkompetenz}

\begin{quote}
»Die erhobenen Daten sind unter der Annahme gleicher
Mathematikkompetenzen von deutschen Schülerinnen und Schülern und dem
OECD-Durchschnitt sehr unwahrscheinlich.«
\end{quote}

Hier wird eine

\begin{itemize}
\item
  \begin{enumerate}
  \def\labelenumi{(\Alph{enumi})}
  \tightlist
  \item
    Deskriptivstatistik\\
  \end{enumerate}
\item
  \begin{enumerate}
  \def\labelenumi{(\Alph{enumi})}
  \setcounter{enumi}{1}
  \tightlist
  \item
    Inferenzstatistik
  \end{enumerate}
\end{itemize}

präsentiert

\subsection*{Standzeiten von
Untertiteln}\label{standzeiten-von-untertiteln}
\addcontentsline{toc}{subsection}{Standzeiten von Untertiteln}

\begin{quote}
»Nach der Betrachtung von gekürzten Untertiteln mit längeren Standzeiten
zeigten 58\% der Kinder ein Verständnis über dem durchschnittlichen
Verständnis von Sendungen mit konventionellen Untertiteln.«
\end{quote}

Hier wird eine

\begin{itemize}
\item
  \begin{enumerate}
  \def\labelenumi{(\Alph{enumi})}
  \tightlist
  \item
    Deskriptivstatistik\\
  \end{enumerate}
\item
  \begin{enumerate}
  \def\labelenumi{(\Alph{enumi})}
  \setcounter{enumi}{1}
  \tightlist
  \item
    Inferenzstatistik
  \end{enumerate}
\end{itemize}

präsentiert

\section{Interpretation}\label{interpretation}

\subsection{Inferenz- und
Deskriptivstatistik}\label{inferenz--und-deskriptivstatistik}

\textbf{Deskriptivstatistiken} machen Aussagen über vorliegende
Datensätze z.B. \emph{»Median aller Noten eines Zeugnisses«}

\textbf{Inferenzstatistiken} machen anhand von Daten Aussagen über
(hypothetische) Mechanismen, die diese Daten erzeugen (Eid, Gollwitzer,
\& Schmitt, 2013) z.B. \emph{»Befürworten von 100 zufällig ausgewählten
Erwachsenen 63 Ziffernnoten in der Grundschule, wie sicher liegt dann
eine Zustimmung (\textgreater{} 50\%) in der
Gesamterwachsenenbevölkerung vor?«}

\begin{figure}[H]

{\centering \includesvg{img/ying yang-4.svg}

}

\caption{Komplementarität von Inferenz und Deskriptivstatistik}

\end{figure}%

\bookmarksetup{startatroot}

\chapter{p-Werte}\label{sec-p-Werte}

Hurz!

\bookmarksetup{startatroot}

\chapter{Bayes Factors}\label{sec-Bayes}

Hurz!

\bookmarksetup{startatroot}

\chapter{Konfidenzintervalle}\label{sec-Konfidenzintervalle}

Hurz!

\bookmarksetup{startatroot}

\chapter{Highest Density Posterior Intervals}\label{sec-Highest}

Hurz!

\bookmarksetup{startatroot}

\chapter*{Literatur}\label{literatur}
\addcontentsline{toc}{chapter}{Literatur}

\markboth{Literatur}{Literatur}

\phantomsection\label{refs}
\begin{CSLReferences}{1}{0}
\bibitem[\citeproctext]{ref-eid2013}
Eid, M., Gollwitzer, M., \& Schmitt, M. (2013). \emph{Statistik Und
{Forschungsmethoden}: {Lehrbuch}. {Mit Online-Materialien}} (3. Aufl.).
{Beltz}.

\end{CSLReferences}



\end{document}
